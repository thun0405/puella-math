分解体

$ax^2+bx+c=0$
これが解けるか?

環の様々な定義

UFD

整域では素元が定義できる

既約元

イデアル
単項イデアル=一つの元で生成されるイデアル
単項イデアル整域=任意のイデアルが単項イデアルになる整域
素イデアル
極大イデアル

多項式環
既約多項式
準同型写像

剰余定理
$f(x)=3x^3+4x+2$, $g(x)=3x^2+1$
体係数に限定しないの珍しい
\begin{proof}
    存在証明
    $\deg{f} < \deg{g}$の時は自明
    $\deg{f} \geq \deg{g}$の時、最高次係数を合わせて逐次消していけば、有限回で終了する。
    いわゆるユークリッドの互除法

    一意性証明
    $f=pg+q=p'g+q'$とする。
    $\deg{q}, \deg{q'} < {g}$より$\deg{q-q'}<\deg{g}$
    $q-q'=(p'-p)g$
    $p\neq p'$ならば$\deg{q-q'}\geq\deg{g}$で矛盾するため$p=p'$
    よって$q=q'$
\end{proof}

体は係数
\begin{proof}
    unitは零因子にならない
\end{proof}

整域係数の多項式環は整域

体係数の多項式環はPID

