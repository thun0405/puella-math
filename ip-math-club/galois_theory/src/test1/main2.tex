\documentclass[11pt]{jsarticle}

\usepackage{amsmath,amssymb,amscd,amsthm,ascmac} %% euler はeulerFont
\usepackage{graphicx}


\usepackage[svgnames]{xcolor}% tikzより前に読み込む必要あり
\usepackage{tikz}
\usepackage{subfiles}


\newtheoremstyle{jplain}% name
{}% space above
{}% space below
{\normalfont}% body font
{}% indent amount
{\bfseries}% theorem head font
{.}% punctuation after theorem head
{4pt}% space after theorem head (default: 5pt)
{\thmname{#1}\thmnumber{#2}\thmnote{\hspace{2pt}(#3)}}% theorem head spec


\theoremstyle{jplain}
\newtheorem{definition}{定義}[section]
\newtheorem*{definition*}{Definition}  %%*がつくと定理番号がつかない
\newtheorem{theorem}[definition]{定理}
\newtheorem{proposition}[definition]{命題}
\newtheorem{corollary}[definition]{系}
\newtheorem{lemma}[definition]{補題}

\DeclareMathOperator{\image}{Im}
\DeclareMathOperator{\kernel}{Ker}
\DeclareMathOperator{\Hom}{Hom}
\DeclareMathOperator{\identity}{id}
\DeclareMathOperator{\GL}{GL}
\DeclareMathOperator{\End}{End}
\DeclareMathOperator{\Sym}{Sym}
\newcommand{\BBBBBBBB}{TestTestTestBBBBBB}

\begin{document}
    分解体

    $ax^2+bx+c=0$
    これが解けるか?

    環の様々な定義

    UFD
%    \subfile{test2/sub1.tex}
    %\documentclass[../main.tex]{subfiles}
\begin{document}
    整域では素元が定義できる
    \AAAAAAAA
    \AAAAAAAA

    既約元

    イデアル
    単項イデアル=一つの元で生成されるイデアル
    単項イデアル整域=任意のイデアルが単項イデアルになる整域
    素イデアル
    極大イデアル
\end{document}
    \AAAAAAAA

    多項式環
    既約多項式
    準同型写像

    剰余定理
    $f(x)=3x^3+4x+2$, $g(x)=3x^2+1$
    体係数に限定しないの珍しい
    \begin{proof}
        存在証明
        $\deg{f} < \deg{g}$の時は自明
        $\deg{f} \geq \deg{g}$の時、最高次係数を合わせて逐次消していけば、有限回で終了する。
        いわゆるユークリッドの互除法

        一意性証明
        $f=pg+q=p'g+q'$とする。
        $\deg{q}, \deg{q'} < {g}$より$\deg{q-q'}<\deg{g}$
        $q-q'=(p'-p)g$
        $p\neq p'$ならば$\deg{q-q'}\geq\deg{g}$で矛盾するため$p=p'$
        よって$q=q'$
    \end{proof}

    体は係数
    \begin{proof}
        unitは零因子にならない
    \end{proof}
\end{document}
