%! section = 1_.1-1_.1-1-a_topological_groups

\begin{definition}[位相群]
    \label{topological_group}
    %! section = 1_.1-1_.1-1-a_topological_groups

$G$を群かつHausdorff空間とする.
次の2条件を満たすとき, $G$を位相群という.

\begin{enumerate}
    \item 写像$G\times G\to G; (x,y)\mapsto xy$は連続である.
    \item 写像$G\to G; x\mapsto x^{-1}$は連続である.
\end{enumerate}

\end{definition}

\begin{proposition}
    $G$を群かつHausdorff空間とする.
    次は同値である.
    \begin{enumerate}
        \item $G$は位相群である.
        \item 写像$G\times G \to G ; (x, y)\mapsto x^{-1}y$は連続である.
        \item 任意の$x,y\in G$と$x^{-1}y$の任意の開近傍$U$に対し,
        $x, y$の開近傍$V, W$が存在し$V^{-1}W\subset U$を満たす.
        ただし, $V^{-1}W= \{x^{-1}y\in G \mid x\in V, y\in W \}$.
    \end{enumerate}
\end{proposition}

\begin{proof}
    $m\colon (x,y)\mapsto xy$,
    $i\colon x\mapsto x^{-1}$,
    $\phi\colon (x,y)\mapsto x^{-1}y$とする.

    (i)$\Rightarrow$(ii):
    $\phi= m\circ (i\times \identity_G)$より$\phi$は連続である.

    (i)$\Leftarrow$(ii):
    $i = \phi(-,e)$より$i$は連続である.
    また, $m= \phi\circ(i\times\identity_G)$より$m$は連続である.

    (ii)$\Rightarrow$(iii):
    仮定より$\phi^{-1}(U)$は開集合であり, かつ$(x,y)\in \phi^{-1}(U)$である.
    したがって$(x, y)$の開近傍$V\times W$で$V\times W \subset \phi^{-1}(U)$であるものが取れる.
    このとき,
    \[V^{-1}W= \phi(V\times W) \subset U
    \]
    より, $V, W$が求める$x, y$の開近傍である.

    (ii)$\Leftarrow$(iiii):
    任意の開集合$U\subset G$に対し, $\phi^{-1}(U)$が開であることを示せば良い.
    そのために, 各点$(x,y)\in \phi^{-1}(U)$に対し$\phi^{-1}(U)$内で開近傍が取れることを示す.

    $x^{-1}y= \phi(x,y)\in U$なので, 仮定より$x, y$の開近傍$V, W$で$V^{-1}W\subset U$なるものが存在する.
    このとき$\phi(V\times W) = V^{-1}W$より
    \[V\times W \subset \phi^{-1}(V^{-1}W) \subset \phi^{-1}(U)
    \]
    であるから, $V\times W$が求める$(x,y)$の開近傍である.
\end{proof}

\begin{proposition}
    %! section = 1_.1-1_.1-1-a_topological_groups

位相群$G$の部分空間$H$が$G$の部分群ならば,
$H$は位相群である.

\end{proposition}

\begin{proof}
    $H$は明らかに$G$のHausdorff性を継承する.

    写像$(x,y)\mapsto x^{-1}y$の連続性を示す.
    任意の$x, y\in H$と$x^{-1}y$の任意の開近傍$U\subset G$に対し,
    $H$内で$x, y$の適切な開近傍が取れれば良い.
    仮定より, $x, y$の開近傍$V, W\subset G$で$V^{-1}W\subset U$なるものが取れる.
    このとき
    \[(V\cap H)^{-1} \cdot W\cap H
    = V^{-1}\cap H \cdot W\cap H
    \subset (V^{-1}W) \cap H \subset U\cap H
    \]
    より, $V\cap H , W\cap H$が求める$x, y$の開近傍である.
\end{proof}

\begin{proposition}
    %! section = 1_.1-1_.1-1-a_topological_groups

$G$を位相群, $H\subset G$を正規部分群とする.
$G/H$を商空間とすると,
自然な写像$\pi\colon G\to G/H$は開写像である.

\end{proposition}

\begin{proof}
    任意の開集合$U\subset G$に対し, $\pi(U)\subset G/H$が開,
    すなわち$\pi^{-1}(\pi(U))$が開であればよい.

    まず,
    \[\pi^{-1}(\pi(U)) = \bigcup_{u\in U}uH = \bigcup_{h\in H}Uh
    \]
    である.
    また, 右移動$G\to G ;x\mapsto xh$は同相であるため,
    各$Uh$は開集合であり,
    したがって$\pi^{-1}(\pi(U))$は開集合である.
\end{proof}

\begin{proposition}
    %! section = 1_.1-1_.1-1-a_topological_groups

$G$を位相群, $H\subset G$を正規部分群とする.
$H$が閉ならば, 商空間$G/H$はHausdorffである
(したがって, このとき$G/H$は位相群をなす).

\end{proposition}

\begin{proof}
    $x, y\in G$について, $xH \neq yH$のとき$x^{-1}y\in G\setminus H$である.
    $H$が閉ならば$G\setminus H$は開であるので,
    写像$(x,y)\mapsto x^{-1}y$の連続性より,
    $x, y$の開近傍$U, V$が存在して$U^{-1}V\subset G\setminus H$を満たす.
    $\pi$は開写像であるから, $\pi(U), \pi(V)$が$xH, yH$を開近傍である.
    また, このとき$\pi(U)\cap \pi(V)= \emptyset$であることが次のように示される.
    $\pi(U)\cap \pi(V)\neq \emptyset$であるとすると,
    ある$u\in U, v\in V$が存在して$u^{-1}v\in H$となるが,
    \[u^{-1}v \in (U^{-1}V)\cap H \subset (G\setminus H)\cap H = \emptyset
    \]
    となり矛盾する.
\end{proof}
上の証明において, $G$のHausdorff性を使わないことに注意.


相互参照テスト\ref{topological_group}
